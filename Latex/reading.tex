\documentclass[11pt]{article}
\usepackage[T1]{fontenc}
\usepackage{fourier}
\usepackage{setspace}
\usepackage[protrusion=true,expansion=true]{microtype}	
\usepackage{amsmath,amsfonts,amsthm} % Math packages
\usepackage[pdftex]{graphicx}	
\usepackage{subfigure}
\usepackage{url}
\usepackage{caption}
\usepackage{float}
\usepackage{fancyhdr}
\usepackage{geometry}
\usepackage{amssymb}
\usepackage{amsmath}
\usepackage{wrapfig}
\usepackage{textcomp}
\usepackage{multirow}
\usepackage{boldline}
\usepackage{upgreek}
\usepackage{array}
\usepackage[table]{xcolor}
\usepackage{graphicx}
\usepackage[utf8]{inputenc}
\usepackage[inline]{enumitem}
\usepackage{chngcntr}
\usepackage{verbatim}
\newcommand\bmmax{2}
\usepackage{bm}
\usepackage{siunitx}
\usepackage{eurosym}
\usepackage{booktabs}
\usepackage{fancyhdr}
\usepackage{titlesec}
\usepackage{textgreek}
\usepackage{fancyhdr}
\graphicspath{ {graph/} }
\usepackage{geometry}

\geometry{left=2cm,right=2cm,top=2cm,bottom=2cm}
\pagestyle{fancy}
\fancyhf{}

\counterwithin{table}{section}
\counterwithin{figure}{section}
\counterwithin{equation}{section}

\makeatletter
% This command ignores the optional argument for itemize and enumerate lists
\newcommand{\inlineitem}[1][]{%
\ifnum\enit@type=\tw@
    {\descriptionlabel{#1}}
  \hspace{\labelsep}
\else
  \ifnum\enit@type=\z@
       \refstepcounter{\@listctr}\fi
    \quad\@itemlabel\hspace{\labelsep}
\fi}
\makeatother

\usepackage{helvet}
\renewcommand{\familydefault}{\sfdefault} 

\renewcommand{\rmdefault}{phv} 
 \renewcommand{\sfdefault}{phv} 
\renewcommand{\headrulewidth}{0.5pt}



\begin{document}
\doublespacing
\pagenumbering{arabic}

\title{Design of Electric and Hybrid Formula Student Car}
\author{Team Charge}
\date{\today}
\maketitle


\tableofcontents
%\setlength{\cftsubsecnumwidth}{4em}% Set length of number width in ToC for \subsection
%\makeatother

\newpage

\lhead{Shuqi Zhang}
\rhead{Reading Notes}

\thispagestyle{fancy} 
\clearpage

\section{Future of Employment}
   Due to the improving in the computerisation on labour market, appearently, the occupations mainly consisting of taks following well-defined procedures that can be easily be performed by sophisticated algorithms.\\

   At the same time, with falling prices of computing, problem-solving skills are becoming relatively productive, explaining the substantial employment growth in occupations involving cognitive tasks where skilled labour has a comparative advantage, as well as the persistent increase to education. Therefore, there are pros and cons of the computerisation.\\

   The growing employment in high-income cognitive jobs and low-income manual occupations.\\

   The susceptibility of jobs to computerisation. By catergorising jobs according to their susceptibility to computerisation, allowing us to examine the future direction of technological change in terms of its impact on the occupational composition of the labour market, but also the number of jobs at risk should these technologies materialise.\\

   O*NET - an online service developed for the US Department of Labor, and it has the advantage of providing more recent information on occupational work activities.\\

   Jobs can be examined wherether they can be offshorable based on 1) the job must be performed at a specific work location; and 2) the job requires face-to-face personal communication. Naturally, the characteristics of occupations that can be offshored are different from the charateristics of occupations that can be automated. However, even the job can't be offshorable, it still can be automated. The extent of computerisation is therefore likely to go beyond that of offshoring.\\

   \subsection{review the literature on the historical relationship between techonological progress and exmployment}

   The balance between job conservation and technological progress therefore, to a large extent, reflects the balance of power in society, and how gains from technological progress are being distributed.\\

   The Industrial Revolution was benefited from 1) the property owning classes became politically dominant in Britain. The diffusion of various manufacturing technologies did not impose a risk to the value of their assets. 2) the inventors, consumers and unskilled factory workers largely benefited from mechanisation.\\

   Eventually, the Industrial Revolution created extra values to the society, benefiting the labour force, including the obsoleted skilled artisans.\\

   

   \subsection{describes recent and expected future technological developments}

   \subsection{methodology}

   \subsection{examine the expected impact of these technological developments on labour market outcomes}

   \subsection{conclusions}



   










\end{document}